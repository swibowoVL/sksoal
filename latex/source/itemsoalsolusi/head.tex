%This is a LaTeX template for homework assignments
\documentclass{article}
%\usepackage[utf8]{inputenc}
\usepackage{enumerate}
\usepackage{amsmath}
\usepackage{tikz}
\usepackage{graphicx}
\usepackage{multicol}
\usepackage{geometry} % Required for adjusting page dimensions
\usepackage{fontspec}
\geometry{
      top=1cm, % Top margin
      bottom=1.5cm, % Bottom margin
      left=2cm, % Left margin
      right=2cm, % Right margin
      %showframe, % Uncomment to show how the type block is set on the page
}
\newfontfamily\java[Scale=2.0,Path=./latex/fonts/,Mapping=tex-text, Color=000000]{tl}
\newfontfamily\arabic[Scale=2.0,Path=./latex/fonts/,Mapping=tex-text, Color=000000]{Geeza}
\newfontfamily\japan[Scale=1.0,Path=./latex/fonts/,Mapping=tex-text, Color=000000]{jp}
\newfontfamily\japantwo[Scale=2.0,Path=./latex/fonts/,Mapping=tex-text, Color=000000]{jp}

\begin{document}
\begin{figure}[h]
\begin{tikzpicture}
\node[inner sep=0pt] (russell) at (2.0,2.0)
    {\includegraphics[width=.16\textwidth]{./latex/image/diy.pdf}};
\draw[gray,  thin] (0,0.1) -- (17.3,.1);
\draw[gray, ultra thick ] (0,0.2) -- (17.3,0.2);

\node (c) at (10.4,3.0) {\large PEMERINTAH DAERAH ISTIMEWA YOGYAKARTA};
\node (c) at (10.4,2.45) {\large DINAS PENDIDIKAN, PEMUDA, DAN OLAHRAGA};
\node (c) at (10.4,2.0) {\large \bf{SMA NEGERI 3 YOGYAKARTA}};
\node (c) at (10.4,1.4) {\small Jl. Yos Sudarso No.7 Yogyakarta Telp. (0274) 512856, 520512 };
\node (c) at (10.2,0.8) {\small Laman: www.sman3-yog.sch.id Kode Pos:55224 };
\end{tikzpicture}
\end{figure}
%Introductory example
\begin{center}
    {\large Kunci Jawaban}
\end{center}
%\subsection*{Exam}

    \begin{enumerate}%starts the numbering


