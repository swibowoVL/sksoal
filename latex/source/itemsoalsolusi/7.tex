\item
Objek bermassa 1.0 Kg dikaitkan pada pegas horizontal. Mula-mula pegas ditarik hingga memanjang 0.10 m, kemudian objek dilepas. Setelah itu dibutuhkan waktu 0.5 s sampai kecepatan balok menjadi nol. Berapa kecepatan maksimum objek.
\begin{description}
    \item[Solusi:]
Setelah objek dilepas, objek tersebut akan mengalami gerak harmonis sederhana.\\
Amplitudo pegas yaitu 0.1 m.\\
Setelah objek dilepas sampai kecepatan jadi nol membentuk $\frac{1}{2}$ getaran. Jadi satu getaran penuh membutuhkan waktu (2)(0.5) s$=$1.0 s. Dengan kata lain periode = 1.0 s. Maka $\omega=\frac{2\pi}{T}=6.28$\\

Jadi posisi benda:
\begin{eqnarray*}
x&=&A\cos(\omega t)\\
&=&(0.10 \mbox{ m})\cos(6.28 t) \\
\\
v&=&\frac{dx}{dt}=-(0.1 6.28 \mbox{ m/s}) \sin(6.28 t) \quad \Rightarrow\quad \textrm{$v_{maks}$ bila $\sin(6.28t)=1$, maka} \\
\\
v_{maks}&=&0.628 \mbox{ m/s}
\end{eqnarray*}
\\[1.5cm]

\end{description}
